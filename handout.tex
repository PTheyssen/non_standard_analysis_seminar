\documentclass[a4paper]{article}
\usepackage[german,ngerman]{babel}
\usepackage[utf8]{inputenc}
\usepackage[T1]{fontenc}
\usepackage[autostyle=true,german=quotes]{csquotes}
\usepackage{amsmath}
\usepackage{amssymb}
\usepackage{amsthm}
\usepackage{oldgerm}

\renewcommand{\labelenumi}{(\roman{enumi})}

\title{Nichtstandard Analysis 1}
\date{21.1.2020}
\author{Klaus Philipp Theyssen}

\begin{document}

\maketitle

%\paragraph{Im Folgenden seien}

%\section{Motivation}
%Leibniz und dann Epsilontik von Weierstraß und Bolzano.\\
%Abraham Robinson konstruiert *R (1961) und gibt damit den infitesimalen Größen eine solide formale Basis. \\
%Motivation durch Differentialquotienten.

\section{Konstruktion von *$\mathbb{R}$}
\paragraph{1.1 Definition} $ R \text{ ist der Ring der Folgen }  a = (a^{(n)})_{n \in \mathbb{N}} \text{ reeller Zahlen} $
\begin{enumerate}
      \item Addition, Subtraktion und Multiplikation Komponentenweise, für $a, b \in R$  
            $$ (a_1 \pm b_1, a_2 \pm b_2,...) \text{ und } (a_1 * b_1, a_2 * b_2,...) $$ 
     % \item Die Reellen Zahlen werden kanonisch in R eingebettet, für $ r \in \mathbb{R} $ 
     %       $$ r \mapsto (r,r,r,...) $$
\end{enumerate}
\smallskip
Wir wollen aus R den Körper *R konstruieren, dafür fehlt uns die Division. 

% \paragraph{1.2 Definition} Ideal

\paragraph{1.3 Definition} Sei D ein Ideal in R, für das gilt:  
      $$ a \in D \iff a^{(n)} = 0 \text{ für fast alle } n \in \mathbb{N}  $$

% \paragraph{1.4 Definition} maximales Ideal

\paragraph{1.5 Satz} Jedes echte Ideal in einem Ring mit Einselement ist in einem maximalen Ideal enthalten. (Zornsches Lemma).


\paragraph{1.6 Definition} Äquivalenzrelation auf R  % TODO (nachrechnen, klar machen)
$$ a \equiv b \text{ mod } M \iff a - b \in M $$

\paragraph{1.7 Satz} $ I \text{ maximal } \iff R/I  \text{ ist ein Körper} $

\paragraph{1.8 Definition}  *$\mathbb{R} = R/M $

\paragraph{1.9 Satz} 
Jede Funktion $f:\mathbb{R}^m \to \mathbb{R} $ lässt sich zu einer 
Funktion *$f:$*$\mathbb{R}^m \to \mathbb{R}$ fortsetzen, sodass sie Eigenschaften im Rahmen 
der Logik 1. Stufe behält.


\paragraph{1.10 Definition}  
$$ U = U_M = \{Z(a): a \in M\} \text{ mit } Z(a) = \{n \in \mathbb{N}: a^{(n)} = 0\} $$
$U $ ist ein $\textit{Filter auf } \mathbb{N} $. Ist $ M $ maximales Ideal so ist $ U_M $ ein $ Ultrafilter $, es gelten: 
\begin{enumerate}
      \item $ \emptyset \notin U $      
      \item $ \mathbb{N} \in U$ 
      \item $ Z_1, Z_2 \in U \Rightarrow Z_1 \cap Z_2 \in U$ 
      \item $ Z \in U, \text{ } Z \subset A \subset \mathbb{N} \Rightarrow A \in U$ 
      \item $ A \subset  \mathbb{N} \Rightarrow A \in U \text{ oder } \mathbb{N} \setminus A \in U$ 
      \item $ A \subset  \mathbb{N}, \text{ } |\mathbb{N} \setminus A| < \infty \Rightarrow A \in U$ 
      \item $a \equiv b \text{ mod } M \iff \{n: a^{(n)} = b^{(n)}\} \in U_M$
\end{enumerate}


\section{Eigenschaften von *R}

\paragraph{2.1 Satz} Die Anordnung $\leqslant$ der reellen Zahlen lässt 
sich zu einer Anordnung von *$\mathbb{R}$ fortsetzen.

\paragraph{2.2 Satz} *$\mathbb{R}$ besitzt ein Element das größer als alle reellen Zahlen ist. 
$$ \forall r \in \mathbb{R} : r \leqslant \omega \text{ mod } M \text{, } \omega = (1,2,3,...,n,n+1,...) $$

\paragraph{2.3 Definition} $ \textfrak{D} = \{a \in \text{*}\mathbb{R}: |a| \leqslant r, \text{ für ein } r \in \mathbb{R}\} $ \\
Ist echter konvexer Teilring von *$\mathbb{R}$, die Elemente von \textfrak{D} nennt man $ endliche $ Größen. \\
Konvexität meint hier:
$$ 0 \leqslant b \leqslant a \in \textfrak{D} \Rightarrow b \in \textfrak{D} $$

\paragraph{2.4 Definition} $ \textfrak{M} = \{a \in \text{*}\mathbb{R}: |a| \leqslant \epsilon, \text{ für alle } \epsilon \in \mathbb{R}^+ \} $

\bigskip
\textfrak{M} ist konvexes Ideal in \textfrak{D}, sodass folgende Eigenschaften erfüllt sind 
\begin{enumerate}
      \item $a,b \in \textfrak{M} \Rightarrow a + b \in \textfrak{M} $ 
      \item $a \in \textfrak{M}, b \in \textfrak{D} \Rightarrow a * b \in \textfrak{M} $ 
      \item $0 \leqslant b \leqslant a \in \textfrak{M} \Rightarrow b \in \textfrak{M} $ 
\end{enumerate}
Die Elemente von \textfrak{M} bezeichnen wir als $ unendliche \text{ } kleine $ oder $ infinitesimale $ Größen. 
Alle anderen Elemente von *$\mathbb{R}$, die nicht in \textfrak{M} oder \textfrak{D} sind bezeichnen wir als $ unendliche $ oder $ infinite $ Größen.

\paragraph{2.5 Definition} $a,b \in \text{*}\mathbb{R} \text{ heißen } benachbart \text{ wenn gilt:} $
$$a \approx b \iff a - b \in \textfrak{M}$$
Das heißt a und b unterscheiden sich nur um eine infinitesimale Größe, $\approx$ ist eine Äquivalenzrelation auf *$\mathbb{R}$

\paragraph{2.6 Satz} $ \textit{Jede endliche Größe } a \in \text{*}\mathbb{R} \textit{ ist zu genau einer reellen Zahl r benachbart.}$
$ \textit{r wird dann als der Standardteil } \text{st}(a) \textit{ von a bezeichnet.} $

\paragraph{2.7 Definition} $ \text{Funktion } f:\mathbb{R} \to \mathbb{R} \text{ ist im Punkt } x \in \mathbb{R} \text{ stetig wenn}\\
\text{für alle }\epsilon \in \mathbb{R}^+, \text{ ein } \delta  \in \mathbb{R}^+ \text{ existiert, sodass für alle } h \in \mathbb{R} \text{ gilt:}$
$$  |h| \leqslant \delta \Rightarrow |f(x + h) - f(x)| \leqslant \epsilon $$ 
\paragraph{2.8 Satz} $ \textit{Die Funktion } f:\mathbb{R} \to \mathbb{R} \textit{ ist im Punkt } x \in \mathbb{R} \textit{ stetig}
\textit{ falls für alle }\\ h \approx 0 \textit{ gilt:} $
$$ \text{*}f(x + h) \approx f(x) $$ 

\end{document}