\documentclass[a4paper]{article}
\usepackage[german,ngerman]{babel}
\usepackage[utf8]{inputenc}
\usepackage[T1]{fontenc}
\usepackage[autostyle=true,german=quotes]{csquotes}
\usepackage{amsmath}
\usepackage{amssymb}
\usepackage{amsthm}
\usepackage{oldgerm}

\renewcommand{\labelenumi}{(\roman{enumi})}

\title{Nichtstandard Analysis 1}
\date{21.1.2020}
\author{Klaus Philipp Theyssen}

\begin{document}

%\maketitle

%\paragraph{Im Folgenden seien}

\paragraph{Geschichte, Tratsch}  
\begin{enumerate}
      \item Newton und Leibniz differntiale verwenden infinitesimale Größen 
      \item 19. Jh. Bolzano verbannt infinitesimale Größen aus Beweisen mit Epsilontik, Limes begriff
      \item 1960er formale Konstruktion von *$\mathbb{R}$ durch Abraham Robinson
\end{enumerate}

%Das Rechnen mit unendlich kleinen und unendlich großen Zahlen wie den Leibnizschen Differentialen
%$ dy $ und $ dx $ in früheren Jahrhunderten der Mathematik und Physik selbstverständlich, 
%aber auch umstritten. Konsitenzfragen wie heute kamen erst später, damals reichte den Mathematikern
%mit solchen Größen zu rechnen und richtige Ergebnisse zu erhalten. \\
%Durch die Epsilontik, den Aufbau der Analysis auf dem Limesbegriff, von Weierstraß wurden diese Größen aus 
%exakten Beweisen verbannt. \\ 
%Abraham Robinson (1961) gibt diesen Größen eine formal korrekte Basis durch Konstruktion von *$\mathbb{R}$.

\paragraph{Vorstellung als Funktion der Zeit} Idee, das die infinitesimalen Größen variabel sind und mit der Zeit ab und zunehmen.
Daher Darstellung als Funktion der Zeit (könnte man auch kontinuierlich machen aber diskret einfacher)

\section{Konstruktion von *$\mathbb{R}$}
\paragraph{Definition 1.1} $ R \text{ ist der Ring der Folgen }  a = (a^{(n)})_{n \in \mathbb{N}} \text{ reeller Zahlen} $
\begin{enumerate}
      \item Addition, Subtraktion und Multiplikation komponentenweise, für $a, b \in R$  
            $$ (a^{(1)} \pm b^{(1)}, a^{(2)} \pm b^{(2)},...) \text{ und } (a^{(1)} * b^{(1)}, a^{(2)} * b^{(2)},...) $$ 
      \item Die Reellen Zahlen werden kanonisch in R eingebettet, für $ r \in \mathbb{R} $ 
            $$ r \mapsto (r,r,r,...) $$
\end{enumerate}
\smallskip
Nun wollen wir aus dem Ring R den Körper *R konstruieren, dafür fehlt uns die Division. 


\paragraph{Definition 1.2} Ideal \\
Es sei $ R $ ein Ring. \\ 
$ \text{Ein} \textit{ Ideal } \text{ in } R \text{ ist eine Teilmenge } I \subseteq R \text{, die bezüglich der Addition 
eine} \\ \text{Untergruppe ist und folgende Eigenschaft hat:} $
$$ \forall x \in I, r \in R: xr \in I $$
\paragraph{Beispiel} Kern eines Ringhomomorphismus $ \varphi:R \to S $ ist ein Ideal. 
$$ \text{ker}\varphi = \{r \in R | \varphi(r) = 0\} \subset R $$
denn ist $ \varphi(r) = 0 $ und $ r^\prime \in R $ dann ist auch
$$ \varphi(r^\prime r) = \varphi(r^\prime) \varphi(r) = 0, \text{ also } r^\prime r \in \text{ker}\varphi $$


\paragraph{Definition 1.3} Sei D das Ideal in R, für das gilt:  
      $$ a \in D \iff a^{(n)} = 0 \text{ für fast alle } n \in \mathbb{N}  $$
\paragraph{Motivation} Wir wählen D so weil wir uns wünschen das zwei Folgen
gleich sind wenn sie ab einer gewissen Stelle nicht mehr voneinander unterscheiden,
d.h. ihre differenz liegt in D.   

\paragraph{Definition 1.4} Äquivalenzrelation auf R  % TODO (nachrechnen, klar machen)
$$ a \equiv b \text{ mod } D \iff a - b \in D $$
\begin{enumerate}
      \item reflexiv: $ 0 \in D  $ da $ D $ additive Untergruppe von $ R $: 
      $$ 0 =  a - a \in D $$
     \item symmetrisch: Wenn $ a - b \in D $ dann:
      $$  0 = (a - b) + (b - a) \in D \Rightarrow b - a \in D $$
     \item transitiv: Wenn $ a - b, b - c \in M $ dann:
      $$ (a - b) + (b - c) = a - c \in D $$
\end{enumerate}

\paragraph{Definition Faktorring} Ist $ (R,+,*) $ ein Ring und $ I $ ein (beidseitiges) Ideal
von $ R $, dann bildet die Menge $ R/I = \{a + I | a \in R\} $ der Äquivalenzklassen modulo
$ I $ mit den Verknüpfungen einen Ring:
\begin{enumerate}
    \item $ (a + I) + (b + I) = (a + b) + I  $ 
    \item $ (a + I) * (b + I) = a * b + I $
\end{enumerate}
Diesen Ring nennt man Faktorring $ R $ modulo $ I $ (oder Restklassenring oder Quotientenring)

\paragraph{Bemerkung} Der Faktorring $ R $ modulo $ D $ lässt uns jetzt zwei Folgen als 
gleich ansehen wenn fast alle ihrer Folgenglieder gleich sind. Allerding ist $ R / D $ kein Körper
dafür brauchen wir ein maximales Ideal. 

\paragraph{Definition 1.5} maximales Ideal \\
Ein Ideal $ I \subset R $ heißt maximales Ideal, wenn zwischen I und R kein weiteres Ideal liegt:
$$ I \subset J \subsetneqq R \Rightarrow I = J$$


\paragraph{Satz 1.6} Jedes echte Ideal $ A $ in einem Ring $ R $ mit Einselement ist in einem maximalen Ideal enthalten. (Zornsches Lemma).
\bigskip

Wir beweisen die Aussage mit dem \textbf{Zornschen Lemma}: Eine Halbordnung in der jede Kette eine obere Schranke hat, besitzt ein maximales Element. 
\smallskip

\textbf{Kette}: Sei $ (a,r) $ eine Halbordnung sind zwei Elemente einer Teilmenge $ b $ von $ a $ mit $ r $ vergleichbar 
so heißt $ b $ eine Kette in $ (a,r) $.

\begin{proof}
      Betrachte die Menge $ X $ aller Ideale $ B $ von $ R $ mit $ A \subseteq B \neq R $. 
      Sie ist wegen $ A \in X $ nicht leer und bzgl. der Inklusion $ \subseteq $ geordnet.

      \bigskip
      Es sei $ K $ eine Kette in $ X $ also:
      $$ B, B' \in K \Rightarrow B \subseteq B' \text{ oder } B' \subseteq B $$ 
      Beh: $ C := \bigcup_{B \in K} B $ ist ein Ideal von R. \\

      \textbf{Beweis C ist additive Untergruppe:} \\
      Es seien $ a, a' \in C $ etwa $ a \in B \in K $ und $ a' \in B' \in K $, wegen 
      $ B \subseteq B' \text{ oder } B' \subseteq B $ folgt $ a - a' \in B' \subseteq C $
      oder $ a - a' \in B \subseteq C $  (Untergruppenkrit. !!!)
     
      \bigskip
      \textbf{Idealeigenschaft} 
      $ r a, a r \in B \subseteq C $ also ist C ein Ideal.

      \bigskip
      Es gilt $ A \subseteq C $, es gilt $ 1 \notin B $ für jedes $ B \in K $, da sonst B bereits der ganze Ring wäre, also 
      muss $ 1 \notin C $ gelten. Also $ C \neq R $. Somit liegt $ C $ in $ K $, ist eine obere Schranke in $ K $.
      Folglich besitz $ (X,\subseteq) $ mit dem Zornschen Lemma ein maximales Ideal $ M $ von $ R $ mit $ A \subseteq M $.
\end{proof}



\paragraph{Satz 1.7} $ I \text{ maximal } \iff R/I  \text{ ist ein Körper} $
\begin{proof}
      $\Rightarrow:$  \\
      Sei $ M \subset R $ maximal und $ a + M \neq 0_{R/M} = M, $ also $ a \notin M$. \\
      $ R = M + (a) $ ($ M \subset M \cup \{a\} \neq R $ Widerspruch zu Maximalität von $ M $) \\
      $$ R = M + (a) = \{m + ba | \text{ } m \in M, b \in R\} $$
      $$ \iff \exists b \in R \text{ und } m \in M \text{ mit } ab + m = 1$$ 
      $$ \iff  \exists b \in R \text{ mit } (a + M)(b + M) = 1 + M = 1_{R/M}$$ 
      Somit ist $ R/M $ ein Körper da jedes Element ein Inverses besitzt:
      $$ (a + M)^{-1} = b + M $$
      $\Leftarrow:$ \\
      Ist $ R/M $ ein Körper und $ a \notin M $ dann git es ein $ b \in R $ mit 
      $$ (a + M)(b + M) = 1_{R/M} $$
      $$ \iff M + (a) = R $$  
      Ist also $ M \subsetneqq I $ dann gilt $ I = R $. 
\end{proof}

\paragraph{Definition 1.8} Nichtstandard Zahlenbereich *$\mathbb{R} = R/M $

\paragraph{Bemerkung} Da jede konstante Folge $ \neq 0 $ in $ R $ invertierbar ist,
wird der Unterkörper $ \mathbb{R} $ von $ R $ bei der Restklassenbildung nicht beeinträchtigt, das heißt
wir finden $ \mathbb{R} $ als isomorphes Bild in $ R/M $ wieder.

\paragraph{Satz 1.9} 
Jede Funktion $f:\mathbb{R}^m \to \mathbb{R} $ lässt sich zu einer 
Funktion \\*$f:$*$\mathbb{R}^m \to \text{*}\mathbb{R}$ fortsetzen, sodass sie Eigenschaften die im Rahmen 
der Logik 1. Stufe ausdrückbar sind behält.
\paragraph{Beispiele} Stetigkeit bleibt erhalten
\begin{proof}
      Definiere zu $f:\mathbb{R}^m \to \mathbb{R} $ eine komponentenweise Fortsetzung $ \overline{f} $ durch
      $$ \overline{f}(a_1,...,a_m) = (f(a^{(1)}_1,...,a^{1}_m), f(a^{(2)}_1,...,a^{2}_m),...) $$
      wobei $ a_i $ für $ 1 \leqslant i \leqslant m $ Folgen aus R sind. \\
      Dann setzen wir 
      $$ \text{*}f(a_1,...,a_m) \equiv \overline{f}(a_1,...,a_m) \text{ mod } M $$
      hier sind die Folgen $ a_1,...,a_m $ Vertreter von gewissen Restklassen in $ R/M $
      Jetzt müssen wir die Wohldefiniertheit zeigen, also das diese Definition nicht von der Wahl 
      der Vertreter in $ R/M $ abhängt. Seien 
      $$ a_1 \equiv b_1,...,a_m \equiv b_m \text{ mod } M $$ 
      dann ist zu zeigen, dass
      $$ \overline{f}(a_1,...,a_m) \equiv \overline{f}(b_1,...,b_m) \text{ mod } M $$
      Zunächst betrachten wir das ganze nur für das Ideal $ D $, und wollen dann den Beweis für beliebige Ideale führen,
      dafür werden wir die Definition des Ultrafilters nutzen. \\
      

      \textbf{Fall $ D $ ist das Ideal:} \\
      $ a \equiv b \text{ mod } D $ bedeutet $ a^{(n)} = b^{(n)} $ für fast alle $ n $ (für alle bis auf endlich viele)
      dann gilt aber auch ffa $ n \in \mathbb{N} $:
      $$ a^{(n)}_1 = b^{(n)}_1 \text{ und }...\text{ und } a^{(n)}_m = b^{(n)}_m $$ 
      also gilt für ffa $ n $ 
      $$  f(a^{(n)}_1,...,a^{(n)}_m) = f(b^{(n)}_1,...,b^{(n)}_m) $$
      woraus folgt, dass $ \overline{f}(a_1,...,a_m) \equiv \overline{f}(b_1,...,b_m) \text{ mod } D $

\end{proof}

\paragraph{Definition 1.10}  Für den Beweis von 1.9 mit beliebigem Ideal: 
$$ U = U_M = \{Z(a): a \in M\} \text{ mit } Z(a) = \{n \in \mathbb{N}: a^{(n)} = 0\}$$
$$ \text{ Anmerkung: } U \subseteq \mathcal{P}(\mathbb{N}) $$
$U $ ist ein $\textit{Filter auf } \mathbb{N} $, es gelten die folgenden Eigenschaften: 
\begin{enumerate}
      \item $ \emptyset \notin U $      
      \item $ \mathbb{N} \in U$ 
      \item $ Z_1, Z_2 \in U \Rightarrow Z_1 \cap Z_2 \in U$ 
      \item $ Z \in U, \text{ } Z \subset A \subset \mathbb{N} \Rightarrow A \in U$ 
\end{enumerate}


\begin{proof} $ \text{ } $
      \newpage
\begin{enumerate}
     \item: Nullfolge ist in $ M $ da M Ideal von $ R $ somit ist $ \mathbb{N} \in U $ 
     \item: siehe (i)  
     \item: Seien $ a $ und $ b $ zwei Folgen in $ R/M $ und $ c = a^2 + b^2 $   
            dann ist $ Z(c) = Z(a) \cap Z(b) $
     \item: Sei $ a \in M $ eine Folge mit $ Z(a) = Z $ und $ r \in R, r \neq 0 $ mit $ A = Z(r) $. 
            Da $ a $ Teilmenge von $ r $ ist hat sie an keiner Stelle eine $ 0 $ an der $ r $ nicht auch eine $ 0 $ hat 
            mit der Ideal Eigenschaft folgt sofort $ ar \in M \Rightarrow Z(ar) = A \in U $  
\end{enumerate}
\end{proof}

\paragraph{Hilfssatz 1.11}  
            $$ a \equiv b \text{ mod } M \iff \{n: a^{(n)} = b^{(n)}\} \in U_M $$
\begin{proof}
      Generell gilt  $$ a \in M \iff Z(a) \in U $$ 
      Wir müssen $ \Leftarrow $ zeigen, ($ \Rightarrow $ ergibt sich wegen der Definition von U).
      Sei $ Z(a) \in U $ wir müssen zeigen, dass $ a \in M $ ist.
      Da $ Z(a) \in U $ ist muss es ein $ b \in M $ geben mit $ Z(a) = Z(b) $ also haben $ a $ und $ b $ 
      die gleichen 0-Stellen. Wir definieren die Folge $ c = (c^{(n)})_{n\in \mathbb{N}} $
      $$ 
      c^{(n)} = 
            \begin{cases}
                  a^{(n)} / b^{(n)} \text{ für } n \notin Z(b) \\
                  0  \text{ sonst}   
            \end{cases}
      $$
      Dann gilt $ a = bc \in M $ \\
      Es gilt $ Z(a -b) = \{n: a^{(n)} = b^{(n)}\} $ somit ergibt sich für $ a,b \in R $ die Eigenschaft
      $$ a \equiv b \text{ mod } M \iff \{n: a^{(n)} = b^{(n)}\} \in U_M $$
\end{proof}

\paragraph{Unabhängigkeitsbeweis für beliebige Ideale}
Es gilt $ a_i \equiv b_i \text{ mod } M $ für $ 1 \leqslant i \leqslant m $ und damit auch $ \{n: a^{(n)}_i = b^{(n)}_i\} \in U $ für 
$ 1 \leqslant i \leqslant m $. Damit gilt auch mit (iii)
$$ \bigcap_{i = 1}^m \{n: a^{(n)}_i = b^{(n)}_i\} = \{n: a^{(n)}_1 = b^{(n)}_1,...,a^{(n)}_m = b^{(n)}_m\} \in U $$
Außerdem gilt, weil eine Funktion (rechtseindeutig): 
$$ \{n: a^{(n)}_1 = b^{(n)}_1,...,a^{(n)}_m = b^{(n)}_m\} \subset \{n: f(a^{(n)}_1,...,a^{(n)}_m) = f(b^{(n)}_1,...,b^{(n)}_m)\} $$
mit (iv) folgt daraus
$$ \{n: f(a^{(n)}_1,...,a^{(n)}_m) = f(b^{(n)}_1,...,b^{(n)}_m)\} \in U $$ 
Mit unserer Folgerung $a \equiv b \text{ mod } M \iff \{n: a^{(n)} = b^{(n)}\} \in U_M$ sind 
$$ \text{*}f(a_1,...,a_m) \equiv \text{*}f(b_1,...,b_m) \text{ mod } M $$ 


\newpage
Ist $ M $ maximales Ideal so ist $ U_M $ ein $ Ultrafilter $, es gelten zusätzlich: 
\begin{enumerate}
      \item[(v)] $ A \subset  \mathbb{N} \Rightarrow A \in U \text{ oder } \mathbb{N} \setminus A \in U$ 
      \item[(vi)] $ A \subset  \mathbb{N}, \text{ } |\mathbb{N} \setminus A| < \infty \Rightarrow A \in U$ 
\end{enumerate}

\begin{proof} $ \text{ } $
\begin{enumerate}
     \item[(v)] Wähle Folge $ a $ aus Nullen und Einsen, so dass $ Z(a) = A $. \\ 
            Angenommen $ A \notin U $ also $ a \notin M $, da M maximal ist existiert $ b \in M $ und $ c \in R $ mit 
            $ 1 = b + ac $. Daher $ Z(b) = Z(1 - ac) \subset \mathbb{N} \setminus A $.  \\ 
            Wegen $ b \in M $ ist $ Z(b) \in U $ mit (iv) erhält man $ \mathbb{N} \setminus A \in U $.
            \bigskip
     \item[(vi)] Wenn $ \mathbb{N} \setminus A < \infty $ dann ist $ A $ unendlich und somit ist die 
            Folge $ a \in R $ aus Nullen und Einsen mit $ Z(a) = A $ im Ideal $ D $.
\end{enumerate}
\end{proof}




\section{Eigenschaften von *R}

\paragraph{Satz 2.2} Die Anordnung $\leqslant$ der reellen Zahlen lässt 
sich zu einer Anordnung von *$\mathbb{R}$ fortsetzen
\begin{proof}
      Sei $ M $ ein maximales Ideal über $ D $, damit ist sichergestellt, dass *$\mathbb{R} = R/M $ ein Körper ist.
      Für $ a,b \in R $ setzen wir 
      $$ a \leqslant b \text{ mod } M :\iff \{n: a^{(n)} \leqslant b^{(n)}\} \in U_M $$ 
      Das bedeutet die Beziehung $ a\leqslant b $ mod $ M $ soll gelten, falls die entsprechende Eigenschaft
      für sehr viele Komponenten gilt. 
      Es bleibt wieder zu zeigen, dass die Definition unabhängig von den Vertretern ist. \\
      Sei also $ a \equiv a_1 $ und $ b \equiv b_1 $ mod $ M $ dann gilt 
      $$ \{n: a^{(n)} \leqslant b^{(n)}\} \cap \{n: a^{(n)} = a^{(n)}_1\} \cap \{n: b^{(n)} = b^{(n)}_1\} \subset \{n: a^{(n)}_1 \leqslant b^{(n)}_1\} $$
      Mit Voraussetzung und mit (iii) ist die linke Seite in $ U $ und dann mit (iv) auch die rechte.

      \bigskip
      Mit den Eigenschaften (v) und (iv) \textit{(brauchen wir um von < zu $ \leqslant $ zu kommen)} von $ U $ folgt für beliebige $ a,b \in R: $
      $$ \{n: a^{(n)} \leqslant b^{(n)}\} \in U \text{ oder } \{n: b^{(n)} \leqslant a^{(n)}\} \in U $$
\end{proof}

\paragraph{\textbf{WEGLASSEN} Eigenschaften Anordnung} Beweise mithilfe der Filter-Eigenschaften von $ U_M $
\begin{enumerate}
     \item $ a \leqslant a $ \\
            Beweis: hier ist $ a^{(n)} = a ^{(n)} $ und damit folgt die schwächere Aussage 
     \item $ a \leqslant b, b \leqslant a \Rightarrow a = b $ \\
            Beweis: $ \forall n $ gilt $ a^{(n)} \leqslant b^{(n)} $ und $ b^{(n)} \leqslant a^{(n)} $ also muss 
            $ \forall n $ gelten $ a^{(n)} = b^{(n)} $ und damit $ a = b $
     \item $ a \leqslant b, b \leqslant c \Rightarrow a \leqslant c $ \\
     \item $ a \leqslant b \Rightarrow a + c \leqslant b + c $ \\ 
     \item $ 0 \leqslant a, 0 \leqslant b \Rightarrow 0 \leqslant ab $ \\

\end{enumerate}


\paragraph{Satz 2.2} *$\mathbb{R}$ besitzt ein Element $ \omega $, das größer als alle reellen Zahlen ist. 
$$ \omega = (1,2,3,...,n,n+1,...)  $$
$$ \forall r \in \mathbb{R} : r \leqslant \omega \text{ mod } M  $$
\begin{proof}
      Wegen (vi) gilt $ \{n: r^{(n)} \leqslant \omega^{(n)} \} = \{n: r \leqslant n\} \in U_M $. \\
      $ \{n: r \leqslant n\} $ hat unendlich viele Elemente, wenn $ r = 0 $ ist, dann $ \{n: r \leqslant n\} = \mathbb{N} $, 
      damit ist $ | \mathbb{N} \setminus A | < \infty $ da r konstant und somit nur endlich viele Folgenglieder größer als $ \omega $  
\end{proof}

\paragraph{Definition 2.3} $ \textfrak{D} = \{a \in \text{*}\mathbb{R}: |a| \leqslant r, \text{ für ein } r \in \mathbb{R}\} $ \\
Ist echter konvexer Teilring von *$\mathbb{R}$, die Elemente von \textfrak{D} nennt man $ endliche $ Größen. \\
Konvexität bedeutet hier:
$$ 0 \leqslant b \leqslant a \in \textfrak{D} \Rightarrow b \in \textfrak{D} $$
\begin{proof}
      Für die Definition haben wir die Fortsetzung *$| \text{ }|$ des Absolutbetrages der reellen Zahlen verwendet. 
      $$
      \text{*}|a| = 
            \begin{cases} 
                  a, \text{ falls } 0 \leqslant a \\
                  -a, \text{ falls } a \leqslant 0
            \end{cases} 
      $$

      \smallskip
      Das die Definition von *$|\text{ }|$ Sinn macht sieht man so ein: Gilt für eine Folge $ 0 \leqslant a $ 
      dann heißt das $ \{n: 0 \leqslant a^{(n)}\} \in U $ dann ist 
      aber auch die Obermenge $ \{n: |a^{(n)}| = a^{(n)}\} $ Element von $ U $. Mit Hilfssatz 1.11 folgt
      $ |a| \equiv a \text{ mod } M $. \\
      ($a \leqslant 0 $ lässt sich analog schließen) \\

      \smallskip
      Die Gleichung und die Ungleichung müsste man modulo $ M $ verstehen, wollen wir aber im folgenden weglassen und 
      $ M $ fest gewählt haben.

\end{proof}


\paragraph{Definition 2.4} $ \textfrak{M} = \{a \in \text{*}\mathbb{R}: |a| \leqslant \epsilon, \text{ für alle } \epsilon \in \mathbb{R}^+ \} $

\bigskip
\textfrak{M} ist konvexes Ideal in \textfrak{D}, sodass folgende Eigenschaften erfüllt sind 
\begin{enumerate}
      \item $a,b \in \textfrak{M} \Rightarrow a + b \in \textfrak{M} $ 
      \item $a \in \textfrak{M}, b \in \textfrak{D} \Rightarrow a * b \in \textfrak{M} $ 
      \item $0 \leqslant b \leqslant a \in \textfrak{M} \Rightarrow b \in \textfrak{M} $ 
\end{enumerate}
\begin{proof}
      \textfrak{M} besteht nicht nur aus der Null, denn wegen $ n \leqslant \omega $ folgt
      $ 0 \leqslant 1/\omega \leqslant 1/n $ für alle $ n \in \mathbb{N} $ also ist $ 1/\omega \in \textfrak{M} $. \\

      \end{proof}

Die Elemente von \textfrak{M} bezeichnen wir als $ unendliche \text{ } kleine $ oder $ infinitesimale $ Größen. 
Alle anderen Elemente von *$\mathbb{R}$, die nicht in \textfrak{M} oder \textfrak{D} sind bezeichnen wir als $ unendliche $ oder $ infinite $ Größen.

\paragraph{Definition 2.5} $a,b \in \text{*}\mathbb{R} \text{ heißen } benachbart \text{ wenn gilt:} $
$$a \approx b \iff a - b \in \textfrak{M}$$
Das heißt a und b unterscheiden sich nur um eine infinitesimale Größe, $\approx$ ist eine Äquivalenzrelation auf *$\mathbb{R}$

\paragraph{Satz 2.6} $ \textit{Jede endliche Größe } a \in \text{*}\mathbb{R} \textit{ ist zu genau einer reellen Zahl r benachbart.}$
$ \textit{r wird dann als der Standardteil } \text{st}(a) \textit{ von a bezeichnet.} $
\begin{proof}
      \textit{Existenz} \\
      Wir betrachten die Mengen $ X_a = \{r \in \mathbb{R}: r \leqslant a\} $ und  
      $ Y_a = \{s \in \mathbb{R}: a \leqslant s\} $ damit definieren $ X_a $ und $ Y_a $ einen Schnitt in $ \mathbb{R} $.
      Wegen der Schnittvollständigkeit \textit{(meint jeder schnitt wird durch reelle Zahl repräsentiert, es gibt keine Lücken in der Zahlenstrahl)} von $ \mathbb{R} $ gibt es ein $ t \in \mathbb{R} $ mit $ r \leqslant t \leqslant s $ 
      für $ r \in X_a, s \in Y_a $. Damit erhält man $ |t - a| \leqslant \epsilon $ für alle $ \epsilon \in \mathbb{R}^+ $.
      Also $ a \approx t \in \mathbb{R} $

      \bigskip
      \textit{Eindeutigkeit} \\
      Angenommen es sei $ t_1 \approx a \approx t_2 $ für $ t_1, t_2 \in \mathbb{R} $ dann folgt $ t_1 \approx t_2 $ 
      das heißt $ |t_1 - t_2| < \epsilon $ für alle $ \epsilon \in \mathbb{R}^+ $. Das ist nur für $ t_1 - t_2 = 0 $ möglich.
\end{proof}

\paragraph{Anmerkung}
$st: \textfrak{D} \to \mathbb{R} $ ist ein ordnungserhaltender Ringhomomorphismus mit Kern \textfrak{M}
und st$|\mathbb{R} =$ id. Daraus folgt das wir mit den endlichen Größen in *$\mathbb{R} $ alle Körperoperationen 
dürchführen können wenn wir $ \approx $ statt $ = $ schreiben und nur durch $ a \in \textfrak{D} $ dividieren 
wenn gilt $ a \not \approx 0 $.


\paragraph{Definition 2.7} $ \text{Stetigkeit einer Funktion } f:\mathbb{R} \to \mathbb{R} \text{ im Punkt } x \in \mathbb{R} \\
\text{seien }\epsilon, \delta  \in \mathbb{R}^+ \text{ und } h \in \mathbb{R} \text{ dann muss gelten:}$
$$ \forall \epsilon \text{ } \exists \delta \textit{sodass für alle h gilt: } |h| \leqslant \delta \Rightarrow |f(x + h) - f(x)| \leqslant \epsilon $$ 

Alternative Definiton:
$$ \forall \epsilon > 0 \exists \delta > 0 \forall x \in \mathbb{R}: (|x - x_0|) < \delta \Rightarrow |f(x) - f(x_0) | < \epsilon)$$
\begin{proof}
      Intuition: Zu jeder Änderung $ \epsilon $ des Funktionswertes, die man fest 
      wählt, kann man eine maximale Änderung $ \delta $ im Argument finden sodass
      die Vorgabe durch $ \epsilon $ eingehalten wird. \\  

      \smallskip
      \textbf{Skizze}
\end{proof}



\paragraph{Satz 2.8} $ \textit{Die Funktion } f:\mathbb{R} \to \mathbb{R} \textit{ ist im Punkt } x \in \mathbb{R} \textit{ stetig }
\iff \\ \textit{ für alle } h \approx 0 \textit{ gilt:} $
$$ \text{*}f(x + h) \approx f(x) $$ 

\paragraph{Bemerkung} Im folgenden Beweis ist es wichtig sich klar zu machen ob man Folgen also Elemente in *$\mathbb{R}$ oder reelle Zahlen 
miteinander betrachtet bzw. vergleichen.

\begin{proof}
      Beweis $ \Rightarrow $: \\
      Mit Satz 1.9 lässt sich $ f $ zu Funktion *$f: \text{*}\mathbb{R} \to \text{*}\mathbb{R} $ fortsetzen sodass für 
      $ a = (a^{(n)})_{n\in\mathbb{N}} $ gilt 
      $$ \text{*}f(a) \equiv (f(a^{(n)}))_{n\in\mathbb{N}} \text{ mod } M $$

      \smallskip
      Wir haben angenommen, dass f stetig ist, sei $ h \in \text{*}\mathbb{R} $ mit $ |h| \leqslant \delta $ \textit{(delta ist eine konstante Folge)}.

      \smallskip
      Mit der Anordnung in *$\mathbb{R}$ gilt $ \{n: |h^{(n)}| \leqslant \delta\} \in U $. 
      
      \smallskip
      Wegen der Stetigkeit von $ f $ gilt
      $ \{n: |h^{(n)}| \leqslant \delta\} \subset \{n: |f(x + h^{(n)}) - f(x)) \leqslant \epsilon\} $ 
      und wegen Filter-Eigenschaft (iv) ist auch die letztere Menge in $ U $, das 
      heißt es gilt $ |\text{*}f(x + h) - \text{*}f(x)| \leqslant \epsilon $. Ist $ h \in \textfrak{M} $ 
      (also $ h \approx 0 $) so ist $ |h| \leqslant \delta $ für alle $ \delta \in \mathbb{R}^+ $ richtig. 

      Damit auch die Folgerung das $ |\text{*}f(x + h) - \text{*}f(x)| \leqslant \epsilon $ für alle
      $ \epsilon \in  \mathbb{R}^+ $ gilt. Somit unterscheiden sich $ \text{*}f(x + h) $ und $  \text{*}f(x) $ nur
      um eine infinitesimale Größe, was heißt:
      $$ \text{*}f(x + h) \approx *f(x) $$

      Beweis $ \Leftarrow $: durch Kontraposition \\
      Wir nehmen also an $ f $ ist in $ x $ nicht stetig. 
      $$ \exists \epsilon \forall \delta \exists h: |h| \leqslant \delta \wedge |f(x + h) - f(x)| > \epsilon $$
      Dann gibt es ein $ \epsilon \in \mathbb{R}^+ $, sodass 
      zu jedem $ n \in \mathbb{N} $ ein $ h^{(n)} \in \mathbb{R} $ existiert mit $ |h^{(n)}| \leqslant \frac{1}{n} $
      und $ |f(x + h^{(n)}) - f(x)| \geq \epsilon $. 

      \smallskip
      Setzen wir nun $ h = (h^{(n)})_{n\in\mathbb{N}} $ so gilt $ |h| \leqslant 1/\omega $ wegen 
      $ 1/\omega \in \textfrak{M} $ gilt auch $ h \in \textfrak{M} $, d.h. $ h \approx 0 $.

      \smallskip
      Wir haben $ h^{(n)} $ so gewählt das gilt $ \mathbb{N} = \{n: |f(x + h^{(n)}) - f(x)| \geq \epsilon \} \in U $,
      das heißt \textit{(wegen Anordnung von }*$\mathbb{R}$) $ |\text{*}f(x + h) - f(x)| \geq \epsilon $ das ist ein Widerspruch zur Vorraussetzung 
      $\text{*}f(x + h) \approx f(x) $ für $ h \approx 0$. 


\end{proof}



\end{document}